\documentclass[11pt, a4paper, twoside]{article}
\usepackage[T1]{fontenc}
\usepackage[utf8]{inputenc}
\usepackage{amssymb,amsmath}
\usepackage[portuguese]{babel}
\usepackage{comment}
\usepackage{datetime}
\usepackage[pdfusetitle]{hyperref}
\usepackage[all]{xy}
\usepackage{graphicx}
\addtolength{\parskip}{.5\baselineskip}

%aqui comeca o que eu fiz de verdade, o resto veio e eu to com medo de tirar
\usepackage{xcolor}
\usepackage{listings} %biblioteca pro codigo
\usepackage{color}    %deixa o codigo colorido bonitinho
\usepackage[landscape, left=0.5cm, right=0.5cm, top=1cm, bottom=1.5cm]{geometry} %pra deixar a margem do jeito que o brasil gosta

\definecolor{gray}{rgb}{0.4, 0.4, 0.4} %cor pros comentarios
%\renewcommand{\footnotesize}{\small} %isso eh pra mudar o tamanho da fonte do codigo
\setlength{\columnseprule}{0.2pt} %barra separando as duas colunas
\setlength{\columnsep}{10pt} %distancia do texto ate a barra

\lstset{ %opcoes pro codigo
breaklines=true,
keywordstyle=\color{blue}\bfseries,
commentstyle=\color{gray},
breakatwhitespace=true,
language=C++,
%frame=single, % nao sei se gosto disso ou nao
numbers=none,
rulecolor=\color{black},
showstringspaces=false
stringstyle=\color{blue},
tabsize=4,
basicstyle=\ttfamily\footnotesize, % fonte
}
\lstset{literate=
%   *{0}{{{\color{red!20!violet}0}}}1
%    {1}{{{\color{red!20!violet}1}}}1
%    {2}{{{\color{red!20!violet}2}}}1
%    {3}{{{\color{red!20!violet}3}}}1
%    {4}{{{\color{red!20!violet}4}}}1
%    {5}{{{\color{red!20!violet}5}}}1
%    {6}{{{\color{red!20!violet}6}}}1
%    {7}{{{\color{red!20!violet}7}}}1
%    {8}{{{\color{red!20!violet}8}}}1
%    {9}{{{\color{red!20!violet}9}}}1
%	 {l}{$\text{l}$}1
	{~}{$\sim$}{1} % ~ bonitinho
}

\title{Manda o Double de Campeão \\ CEFET-MG}
\author{Pedro Augusto}


\begin{document}
\twocolumn
\date{\today}
\maketitle


\renewcommand{\contentsname}{Índice} %troca o nome do indice para indice
\tableofcontents


%%%%%%%%%%%%%%%%%%%%
%
% Estruturas
%
%%%%%%%%%%%%%%%%%%%%

\section{Estruturas}

\subsection{Fenwick Tree (BIT)}
\begin{lstlisting}
// Operacoes 0-based
// query(l, r) retorna a soma de v[l..r]
// update(l, r, x) soma x em v[l..r]
//
// Complexidades:
// build - O(n)
// query - O(log(n))
// update - O(log(n))

e04 namespace bit {
06d 	int bit[2][MAX+2];
1a8 	int n;
    
727 	void build(int n2, vector<int>& v) {
1e3 		n = n2;
535 		for (int i = 1; i <= n; i++)
a6e 			bit[1][min(n+1, i+(i&-i))] += bit[1][i] += v[i];
d31 	}
1a7 	int get(int x, int i) {
7c9 		int ret = 0;
360 		for (; i; i -= i&-i) ret += bit[x][i];
edf 		return ret;
a4e 	}
920 	void add(int x, int i, int val) {
503 		for (; i <= n; i += i&-i) bit[x][i] += val;
fae 	}
3d9 	int get2(int p) {
c7c 		return get(0, p) * p + get(1, p);
33c 	}
9e3 	int query(int l, int r) { // zero-based
ff5 		return get2(r+1) - get2(l);
25e 	}
7ff 	void update(int l, int r, int x) {
e5f 		add(0, l+1, x), add(0, r+2, -x);
f58 		add(1, l+1, -x*l), add(1, r+2, x*(r+1));
5ce 	}
17a };

63d void solve() {
    
97a 	vector<int> v {0,1,2,3,4,5}; // v[0] eh inutilizada
c7b 	bit::build(v.size(), v);
    
67f 	int a = 0, b = 3;
9b0 	bit::query(a, b); // v[a] + v[a+1] + ... + v[b] = 6  | 1+2+3 = 6 | zero-based
b3d 	bit::update(a, b, 2); // v[a...b] += 2 | zero-based
7b4 }
\end{lstlisting}

\subsection{SegTree}
\begin{lstlisting}
// Recursiva com Lazy Propagation
// Query: soma do range [a, b]
// Update: soma x em cada elemento do range [a, b]
// Pode usar a seguinte funcao para indexar os nohs:
// f(l, r) = (l+r)|(l!=r), usando 2N de memoria
//
// Complexidades:
// build - O(n)
// query - O(log(n))
// update - O(log(n))

0d2 const int MAX = 1e5+10;

fb1 namespace SegTree {
098 	int seg[4*MAX], lazy[4*MAX];
052 	int n, *v;
    
b90 	int op(int a, int b) { return a + b; }
    
2c4 	int build(int p=1, int l=0, int r=n-1) {
3c7 		lazy[p] = 0;
6cd 		if (l == r) return seg[p] = v[l];
ee4 		int m = (l+r)/2;
317 		return seg[p] = op(build(2*p, l, m), build(2*p+1, m+1, r));
985 	}
    
0d8 	void build(int n2, int* v2) {
680 		n = n2, v = v2;
6f2 		build();
acb 	}
    
ceb 	void prop(int p, int l, int r) {
cdf 		seg[p] += lazy[p]*(r-l+1);
2c9 		if (l != r) lazy[2*p] += lazy[p], lazy[2*p+1] += lazy[p];
3c7 		lazy[p] = 0;
c10 	}
    
04a 	int query(int a, int b, int p=1, int l=0, int r=n-1) {
6b9 		prop(p, l, r);
527 		if (a <= l and r <= b) return seg[p];
786 		if (b < l or r < a) return 0;
ee4 		int m = (l+r)/2;
19e 		return op(query(a, b, 2*p, l, m), query(a, b, 2*p+1, m+1, r));
1c9 	}
    	
f33 	int update(int a, int b, int x, int p=1, int l=0, int r=n-1) {
6b9 		prop(p, l, r);
9a3 		if (a <= l and r <= b) {
b94 			lazy[p] += x;
6b9 			prop(p, l, r);
534 			return seg[p];
821 		}
e9f 		if (b < l or r < a) return seg[p];
ee4 		int m = (l+r)/2;
a8f 		return seg[p] = op(update(a, b, x, 2*p, l, m), update(a, b, x, 2*p+1, m+1, r));
08f 	}
    
    	// Se tiver uma seg de max, da pra descobrir em O(log(n))
    	// o primeiro e ultimo elemento >= val numa range:
    
    	// primeira posicao >= val em [a, b] (ou -1 se nao tem)
119 	int get_left(int a, int b, int val, int p=1, int l=0, int r=n-1) {
6b9 		prop(p, l, r);
f38 		if (b < l or r < a or seg[p] < val) return -1;
205 		if (r == l) return l;
ee4 		int m = (l+r)/2;
753 		int x = get_left(a, b, val, 2*p, l, m);
50e 		if (x != -1) return x;
c3c 		return get_left(a, b, val, 2*p+1, m+1, r);
68c 	}
    
    	// ultima posicao >= val em [a, b] (ou -1 se nao tem)
992 	int get_right(int a, int b, int val, int p=1, int l=0, int r=n-1) {
6b9 		prop(p, l, r);
f38 		if (b < l or r < a or seg[p] < val) return -1;
205 		if (r == l) return l;
ee4 		int m = (l+r)/2;
1b1 		int x = get_right(a, b, val, 2*p+1, m+1, r);
50e 		if (x != -1) return x;
6a7 		return get_right(a, b, val, 2*p, l, m);
1b7 	}
    
    	// Se tiver uma seg de soma sobre um array nao negativo v, da pra
    	// descobrir em O(log(n)) o maior j tal que v[i]+v[i+1]+...+v[j-1] < val
89b 	int lower_bound(int i, int& val, int p, int l, int r) {
6b9 		prop(p, l, r);
6e8 		if (r < i) return n;
b5d 		if (i <= l and seg[p] < val) {
bff 			val -= seg[p];
041 			return n;
634 		}
3ce 		if (l == r) return l;
ee4 		int m = (l+r)/2;
514 		int x = lower_bound(i, val, 2*p, l, m);
ee0 		if (x != n) return x;
8b9 		return lower_bound(i, val, 2*p+1, m+1, r);
01d 	}
a15 };

63d void solve() {
213 	int n = 10;
89e 	int v[] = {1, 2, 3, 4, 5, 6, 7, 8, 9, 10};
2d5 	SegTree::build(n, v);
    
3af 	cout << SegTree::query(0, 9) << endl; // seg[0] + seg[1] + ... + seg[9] = 55
310 	SegTree::update(0, 9, 1); // seg[0,...,9] += 1
6d9 }
\end{lstlisting}

\subsection{Sparse Table Disjunta}
\begin{lstlisting}
// Description: Sparse Table Disjunta para soma de intervalos
// Complexity Temporal: O(n log n) para construir e O(1) para consultar
// Complexidade Espacial: O(n log n)

2b7 #include <bits/stdc++.h>
ca4 using namespace std;

005 #define MAX 100010
352 #define MAX2 20 // log(MAX)

82d namespace SparseTable {
9bf 	int m[MAX2][2*MAX], n, v[2*MAX];
b90 	int op(int a, int b) { return a + b; }
0d8 	void build(int n2, int* v2) {
1e3 		n = n2;
df4 		for (int i = 0; i < n; i++) v[i] = v2[i];
a84 		while (n&(n-1)) n++;
3d2 		for (int j = 0; (1<<j) < n; j++) {
1c0 			int len = 1<<j;
d9b 			for (int c = len; c < n; c += 2*len) {
332 				m[j][c] = v[c], m[j][c-1] = v[c-1];
668 				for (int i = c+1; i <  c+len; i++) m[j][i] = op(m[j][i-1], v[i]);
432 				for (int i = c-2; i >= c-len; i--) m[j][i] = op(v[i], m[j][i+1]);
eda 			}
f4d 		}
ce3 	}
9e3 	int query(int l, int r) {
f13 		if (l == r) return v[l];
e6d 		int j = __builtin_clz(1) - __builtin_clz(l^r);
d67 		return op(m[j][l], m[j][r]);
a7b 	}
258 }

63d void solve() {
ce1 	int n = 9;
1a3 	int v[] = {1, 2, 3, 4, 5, 6, 7, 8, 9};
3f7 	SparseTable::build(n, v);
925 	cout << SparseTable::query(0, n-1) << endl; // sparse[0] + sparse[1] + ... + sparse[n-1] = 45
241 }
\end{lstlisting}

\subsection{Tabuleiro}
\begin{lstlisting}
// Description: Estrutura que simula um tabuleiro M x N, sem realmente criar uma matriz
// Permite atribuir valores a linhas e colunas, e consultar a posicao mais frequente
// Complexidade Atribuir: O(log(N))
// Complexidade Consulta: O(log(N))
// Complexidade verificar frequencia geral: O(N * log(N))
9a0 #define MAX_VAL 5 // maior valor que pode ser adicionado na matriz + 1

8ee class BinTree {
d9d     protected:
ef9         vector<int> mBin;
673     public:
d5e         explicit BinTree(int n) { mBin = vector(n + 1, 0); }
    
e44         void add(int p, const int val) {
dd1             for (auto size = mBin.size(); p < size; p += p & -p)
174                 mBin[p] += val;
b68         }
    
e6b         int query(int p) {
e1c             int sumToP {0};
b62             for (; p > 0; p -= p & -p)
ec1                 sumToP += mBin[p];
838             return sumToP;
793         }
a5f };

b6a class ReverseBinTree : public BinTree {
673     public:
83e         explicit ReverseBinTree(int n) : BinTree(n) {};
    
e44         void add(int p, const int val) {
850             BinTree::add(static_cast<int>(mBin.size()) - p, val);
705         }
    
e6b         int query(int p) {
164             return BinTree::query(static_cast<int>(mBin.size()) - p);
a21         }
6cf };

952 class Tabuleiro {
673     public:
177         explicit Tabuleiro(const int m, const int n, const int q) : mM(m), mN(n), mQ(q) {
958             mLinhas = vector<pair<int, int8_t>>(m, {0, 0});
d68             mColunas = vector<pair<int, int8_t>>(n, {0, 0});
    
66e             mAtribuicoesLinhas = vector(MAX_VAL, ReverseBinTree(mQ)); // aARvore[51]
9e5             mAtribuicoesColunas = vector(MAX_VAL, ReverseBinTree(mQ));
13b         }
    
bc2         void atribuirLinha(const int x, const int8_t r) {
e88             mAtribuirFileira(x, r, mLinhas, mAtribuicoesLinhas);
062         }
    
ca2         void atribuirColuna(const int x, const int8_t r) {
689             mAtribuirFileira(x, r, mColunas, mAtribuicoesColunas);
a40         }
    
d10         int maxPosLinha(const int x) {
f95             return mMaxPosFileira(x, mLinhas, mAtribuicoesColunas, mM);
8ba         }
    
ff7         int maxPosColuna(const int x) {
b95             return mMaxPosFileira(x, mColunas, mAtribuicoesLinhas, mN);
252         }
    
80e         vector<int> frequenciaElementos() {
    
a35             vector<int> frequenciaGlobal(MAX_VAL, 0);
45a             for(int i=0; i<mM; i++) {
ebd                 vector<int> curr = frequenciaElementos(i, mAtribuicoesColunas);
97f                 for(int j=0; j<MAX_VAL; j++)
ef3                     frequenciaGlobal[j] += curr[j];
094             }
01e             return frequenciaGlobal;
b7a         }
    
bf2     private:
69d         int mM, mN, mQ, mMoment {0};
    
0a6         vector<ReverseBinTree> mAtribuicoesLinhas, mAtribuicoesColunas;
f2d         vector<pair<int, int8_t>> mLinhas, mColunas;
    
e7a         void mAtribuirFileira(const int x, const int8_t r, vector<pair<int, int8_t>>& fileiras,
1d7                             vector<ReverseBinTree>& atribuicoes) {
224             if (auto& [oldQ, oldR] = fileiras[x]; oldQ)
bda                 atribuicoes[oldR].add(oldQ, -1);
    
914             const int currentMoment = ++mMoment;
b2c             fileiras[x].first = currentMoment;
80b             fileiras[x].second = r;
f65             atribuicoes[r].add(currentMoment, 1);
5de         }
    
2b8         int mMaxPosFileira(const int x, const vector<pair<int, int8_t>>& fileiras, vector<ReverseBinTree>& atribuicoesPerpendiculares, const int& currM) const {
1aa             auto [momentoAtribuicaoFileira, rFileira] = fileiras[x];
    
8d0             vector<int> fileiraFrequencia(MAX_VAL, 0);
729             fileiraFrequencia[rFileira] = currM;
    
85a             for (int8_t r {0}; r < MAX_VAL; ++r) {
8ca                 const int frequenciaR = atribuicoesPerpendiculares[r].query(momentoAtribuicaoFileira + 1);
04a                 fileiraFrequencia[rFileira] -= frequenciaR;
72e                 fileiraFrequencia[r] += frequenciaR;
6b0             }
    
b59             return MAX_VAL - 1 - (max_element(fileiraFrequencia.crbegin(), fileiraFrequencia.crend()) - fileiraFrequencia.crbegin());
372         }
    
7c4         vector<int> frequenciaElementos(int x, vector<ReverseBinTree>& atribuicoesPerpendiculares) const {
                
8d0             vector<int> fileiraFrequencia(MAX_VAL, 0);
    
583             auto [momentoAtribuicaoFileira, rFileira] = mLinhas[x];
    
083             fileiraFrequencia[rFileira] = mN;
    
85a             for (int8_t r {0}; r < MAX_VAL; ++r) {
8ca                 const int frequenciaR = atribuicoesPerpendiculares[r].query(momentoAtribuicaoFileira + 1);
04a                 fileiraFrequencia[rFileira] -= frequenciaR;
72e                 fileiraFrequencia[r] += frequenciaR;
6b0             }
    
2e6             return fileiraFrequencia;
15d         }
    
20c };

63d void solve() {
    
e29     int L, C, q; cin >> L >> C >> q;
    
56c     Tabuleiro tabuleiro(L, C, q);
    
a09     int linha = 0, coluna = 0, valor = 10; // linha e coluna sao 0 based
b68     tabuleiro.atribuirLinha(linha, static_cast<int8_t>(valor)); // f(i,0,C) matriz[linha][i] = valor
34d     tabuleiro.atribuirColuna(coluna, static_cast<int8_t>(valor)); // f(i,0,L) matriz[i][coluna] = valor
    
        // Freuencia de todos os elementos, de 0 a MAX_VAL-1
155     vector<int> frequenciaGeral = tabuleiro.frequenciaElementos();
    
176     int a = tabuleiro.maxPosLinha(linha); // retorna a posicao do elemento mais frequente na linha
981     int b = tabuleiro.maxPosColuna(coluna); // retorna a posicao do elemento mais frequente na coluna
9b5 }
\end{lstlisting}

\subsection{Union-Find (Disjoint Set Union)}
\begin{lstlisting}
f3b const int MAX = 5e4+10;

074 int p[MAX], ranking[MAX], setSize[MAX];

0cd struct UnionFind {
c55     int numSets;
    
02d     UnionFind(int N) {
680 		iota(p,p+N+1,0);
340 		memset(ranking, 0, sizeof ranking);
f0a 		memset(setSize, 1, sizeof setSize);
0bd         numSets = N;
142     }
    
c59     int numDisjointSets() { return numSets; }
a5b     int sizeOfSet(int i) { return setSize[find(i)]; }
    
8ee     int find(int i) { return (p[i] == i) ? i : (p[i] = find(p[i])); }
da3     bool same(int i, int j) { return find(i) == find(j); }
92e     void uni(int i, int j) {
ea5         if (same(i, j))
505             return;
c56         int x = find(i), y = find(j);
e4f         if (ranking[x] > ranking[y])
9dd             swap(x, y);
ae9         p[x] = y;
6e9         if (ranking[x] == ranking[y])
3cf             ++ranking[y];
223         setSize[y] += setSize[x];
92a         --numSets;
e3f     }
b6b };

63d void solve() {
    
f98 	int n, ed; cin >> n >> ed;
f4e 	UnionFind uni(n);
    
31c 	f(i,0,ed) {
602 		int a, b; cin >> a >> b; a--, b--;
45e 		uni.uni(a,b);
c0f 	}
    
350 	cout << uni.numDisjointSets() << endl;
01b }
\end{lstlisting}

\pagebreak


%%%%%%%%%%%%%%%%%%%%
%
% Extra
%
%%%%%%%%%%%%%%%%%%%%

\section{Extra}

\subsection{fastIO.cpp}
\begin{lstlisting}
int read_int() {
    bool minus = false;
    int result = 0;
    char ch;
    ch = getchar();
    while (1) {
        if (ch == '-') break;
        if (ch >= '0' && ch <= '9') break;
        ch = getchar();
    }
    if (ch == '-') minus = true;
    else result = ch-'0';
    while (1) {
        ch = getchar();
        if (ch < '0' || ch > '9') break;
        result = result*10 + (ch - '0');
    }
    if (minus) return -result;
    else return result;
}
\end{lstlisting}

\subsection{hash.sh}
\begin{lstlisting}
# Para usar (hash das linhas [l1, l2]):
# bash hash.sh arquivo.cpp l1 l2
sed -n $2','$3' p' $1 | sed '/^#w/d' | cpp -dD -P -fpreprocessed | tr -d '[:space:]' | md5sum | cut -c-6
\end{lstlisting}

\subsection{template.cpp}
\begin{lstlisting}
#include <bits/stdc++.h>
using namespace std;

#define _ ios_base::sync_with_stdio(0);cin.tie(0);
#define all(a)       a.begin(), a.end()
#define int          long long int
#define double       long double
#define f(i,s,e) 	 for(int i=s;i<e;i++)
#define dbg(x) cout << #x << " = " << x << " ";
#define dbgl(x) cout << #x << " = " << x << endl;

#define vi 			 vector<int>
#define pii	         pair<int,int>
#define endl         "\n"
#define print_v(a)   for(auto x : a)cout<<x<<" ";cout<<endl
#define print_vp(a)  for(auto x : a)cout<<x.first<<" "<<x.second<< endl
#define rf(i,e,s) 	 for(int i=e-1;i>=s;i--)
#define CEIL(a, b)   ((a) + (b - 1))/b
#define TRUNC(x, n)  floor(x * pow(10, n))/pow(10, n)
#define ROUND(x, n)  round(x * pow(10, n))/pow(10, n)

const int INF = 1e9;    // 2^31-1
const int LLINF = 4e18; // 2^63-1
const double EPS = 1e-9;
const int MAX = 1e6+10; // 10^6 + 10

void solve() {

	
}

int32_t main() { _
	
	int t = 1; // cin >> t;
	while (t--) {
		solve();
	}
	return 0;
}
\end{lstlisting}

\subsection{stress.sh}
\begin{lstlisting}
P=a
make ${P} ${P}2 gen || exit 1
for ((i = 1; ; i++)) do
	./gen $i > in
	./${P} < in > out
	./${P}2 < in > out2
	if (! cmp -s out out2) then
		echo "--> entrada:"
		cat in
		echo "--> saida1:"
		cat out
		echo "--> saida2:"
		cat out2
		break;
	fi
	echo $i
done
\end{lstlisting}

\subsection{pragma.cpp}
\begin{lstlisting}
// Otimizacoes agressivas, pode deixar mais rapido ou mais devagar
#pragma GCC optimize("Ofast")
// Auto explicativo
#pragma GCC optimize("unroll-loops")
// Vetorizacao
#pragma GCC target("avx2")
// Para operacoes com bits
#pragma GCC target("bmi,bmi2,popcnt,lzcnt")
\end{lstlisting}

\subsection{timer.cpp}
\begin{lstlisting}
// timer T; T() -> retorna o tempo em ms desde que declarou
using namespace chrono;
struct timer : high_resolution_clock {
	const time_point start;
	timer(): start(now()) {}
	int operator()() {
		return duration_cast<milliseconds>(now() - start).count();
	}
};
\end{lstlisting}

\subsection{vimrc}
\begin{lstlisting}
189 "" {
d79 set ts=4 sw=4 mouse=a nu ai si undofile
7c9 function H(l)
496 	return system("sed '/^#/d' | cpp -dD -P -fpreprocessed | tr -d '[:space:]' | md5sum", a:l)
0be endfunction
329 function P() range
dd8 	for i in range(a:firstline, a:lastline)
ccc 		let l = getline(i)
139 		call cursor(i, len(l))
7c9 		echo H(getline(search('{}'[1], 'bc', i) ? searchpair('{', '', '}', 'bn') : i, i))[0:2] l
bf9 	endfor
0be endfunction
90e vmap <C-H> :call P()<CR>
de2 "" }
\end{lstlisting}

\subsection{debug.cpp}
\begin{lstlisting}
void debug_out(string s, int line) { cerr << endl; }
template<typename H, typename... T>
void debug_out(string s, int line, H h, T... t) {
    if (s[0] != ',') cerr << "Line(" << line << ") ";
    do { cerr << s[0]; s = s.substr(1);
    } while (s.size() and s[0] != ',');
    cerr << " = " << h;
    debug_out(s, line, t...);
}
#ifdef DEBUG
#define debug(...) debug_out(#__VA_ARGS__, __LINE__, __VA_ARGS__)
#else
#define debug(...) 42
#endif
\end{lstlisting}

\subsection{makefile}
\begin{lstlisting}
CXX = g++
CXXFLAGS = -fsanitize=address,undefined -fno-omit-frame-pointer -g -Wall -Wshadow -std=c++17 -Wno-unused-result -Wno-sign-compare -Wno-char-subscripts #-fuse-ld=gold

clearexe:
	find . -maxdepth 1 -type f -executable -exec rm {} +
\end{lstlisting}

\subsection{rand.cpp}
\begin{lstlisting}
mt19937 rng((int) chrono::steady_clock::now().time_since_epoch().count());

int uniform(int l, int r){
	uniform_int_distribution<int> uid(l, r);
	return uid(rng);
}
\end{lstlisting}

\end{document}

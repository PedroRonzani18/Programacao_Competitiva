\documentclass[11pt, a4paper, twoside]{article}
\usepackage[T1]{fontenc}
\usepackage[utf8]{inputenc}
\usepackage{amssymb,amsmath}
\usepackage[portuguese]{babel}
\usepackage{comment}
\usepackage{datetime}
\usepackage[pdfusetitle]{hyperref}
\usepackage[all]{xy}
\usepackage{graphicx}
\addtolength{\parskip}{.5\baselineskip}

%aqui comeca o que eu fiz de verdade, o resto veio e eu to com medo de tirar
\usepackage{xcolor}
\usepackage{listings} %biblioteca pro codigo
\usepackage{color}    %deixa o codigo colorido bonitinho
\usepackage[landscape, left=0.5cm, right=0.5cm, top=1cm, bottom=1.5cm]{geometry} %pra deixar a margem do jeito que o brasil gosta

\definecolor{gray}{rgb}{0.4, 0.4, 0.4} %cor pros comentarios
%\renewcommand{\footnotesize}{\small} %isso eh pra mudar o tamanho da fonte do codigo
\setlength{\columnseprule}{0.2pt} %barra separando as duas colunas
\setlength{\columnsep}{10pt} %distancia do texto ate a barra

\lstset{ %opcoes pro codigo
breaklines=true,
keywordstyle=\color{blue}\bfseries,
commentstyle=\color{gray},
breakatwhitespace=true,
language=C++,
%frame=single, % nao sei se gosto disso ou nao
numbers=none,
rulecolor=\color{black},
showstringspaces=false
stringstyle=\color{blue},
tabsize=4,
basicstyle=\ttfamily\footnotesize, % fonte
}
\lstset{literate=
%   *{0}{{{\color{red!20!violet}0}}}1
%    {1}{{{\color{red!20!violet}1}}}1
%    {2}{{{\color{red!20!violet}2}}}1
%    {3}{{{\color{red!20!violet}3}}}1
%    {4}{{{\color{red!20!violet}4}}}1
%    {5}{{{\color{red!20!violet}5}}}1
%    {6}{{{\color{red!20!violet}6}}}1
%    {7}{{{\color{red!20!violet}7}}}1
%    {8}{{{\color{red!20!violet}8}}}1
%    {9}{{{\color{red!20!violet}9}}}1
%	 {l}{$\text{l}$}1
	{~}{$\sim$}{1} % ~ bonitinho
}

\title{Manda o Double de Campeão \\ CEFET-MG}
\author{Pedro Augusto}


\begin{document}
\twocolumn
\date{\today}
\maketitle


\renewcommand{\contentsname}{Índice} %troca o nome do indice para indice
\tableofcontents


%%%%%%%%%%%%%%%%%%%%
%
% Array
%
%%%%%%%%%%%%%%%%%%%%

\section{Array}

\subsection{Longest Increasing Subsequence}
\begin{lstlisting}
// Retorna a mauir subsequencia crescente dentro de um vetor
// O(n logn)
d7d vector<int> lis(vector<int>& arr) {
61d     vector<int> subseq;
ed5     for(int& x : arr) {
8a2         auto it = lower_bound(subseq.begin(), subseq.end(), x);
d3e         if (it == subseq.end()) subseq.push_back(x);
77c         else *it = x;
b53     }
cff     return subseq;
c0e }
\end{lstlisting}



%%%%%%%%%%%%%%%%%%%%
%
% DP
%
%%%%%%%%%%%%%%%%%%%%

\section{DP}

\subsection{Exemplo Sapo}
\begin{lstlisting}
// There are N stones, numbered 1,2,...,N. 
// For each i (1<=i<=N), the height of Stone i is hi.
// There is a frog who is initially on Stone 1. 
// He will repeat the following action some number of times to reach Stone N:
// If the frog is currently on Stone i, jump to one of the following: Stone i+1,i+2,...,i+K. Here, a cost of|hi - hj| is incurred, wherej is the stone to land on.
// Find the minimum possible total cost incurred before the frog reaches Stone N.

dca int n, k;

// Top Down
4d3 int dp(int i) {
563 	if(i == 0) return 0;
7f9 	auto& ans = memo[i];
d64 	if(~ans) return ans;
    
5f9 	int ret = INF;
a7f 	f(j, max(0ll,i-k), i)
f97 		ret = min(ret, dp(j) + abs(h[j] - h[i]));
    
655 	return ans = ret;
641 }

// Bootom Up
e63 int dp_2(int x) {
    	
ecd 	memo[0] = 0;
b85 	f(i,1,x) {
90b 		int best = INF;
203 		f(j, max(0ll, i-k), i) {
428 			best = min(best, memo[j] + abs(h[i] - h[j]));
8b8 		}
bc2 		memo[i] = best;
832 	}
    
d56 	return memo[x-1];
6f3 }

63d void solve() {
0a1 	cin >> n >> k;
3f0 	f(i,0,n) cin >> h[i];
8e4 	cout << dp(n-1) << endl;
1d6 }
\end{lstlisting}

\subsection{Is Subset Sum (Iterativo)}
\begin{lstlisting}
// Verifica se a soma de 0 <= i <= n elementos iguala a sum
// Temporal: O(sum * n)
// Espacial: O(sum * n)

c00 const int MAXN = 100;
bc4 const int MAXSUM = 5000;

759 bool isSubsetSum(vector<int>& v, int n, int sum) {
10a     f(i, 0, n + 1) { memo[i][0] = true; }
258     f(j, 1, sum + 1) { memo[0][j] = false; }
        
336     f(i, 1, n + 1) {
9e0         f(j, 1, sum + 1) {
a1d             if(j < v[i-1])
2b7                 memo[i][j] = memo[i-1][j];
295             else
c1f                 memo[i][j] = memo[i-1][j] || memo[i-1][j- v[i-1]];
66a         }
7f7     }
138     return memo[n][sum];
f54 }

c0b void solve(int n, int sum) {
        
70a     vector<int> v(n);
9b4 	for(auto& x : n) cin >> x;
        
dbf     cout << (isSubsetSum(v, n, k) ? "S" : "N") << endl;
707 }

\end{lstlisting}

\subsection{Knapsack tradicional}
\begin{lstlisting}
// O(n * cap)

b94 const int MAXN = 110;
689 const int MAXW = 1e5+10;

ba9 int n, memo[MAXN][MAXW];
310 int v[MAXN], w[MAXN];
74a int pego[MAXN] = {0};

// Retorna o lucro maximo 
12c int dp(int id, int cap) {
1bb 	if(cap < 0) return -LLINF;
ecb 	if(id == n or cap == 0) return 0;
c1a 	int &ans = memo[id][cap];
d64 	if(~ans) return ans;
86f 	return ans = max(dp(id+1, cap), dp(id+1, cap-w[id]) + v[id]);
d95 }

// Armazena em pego os itens pegos 
7d0 void recuperar(int id, int cap) {
efa 	if(id >= n) return;
fca 	if(dp(id+1, cap-w[id]) + v[id] > dp(id+1, cap)) { // se pegar eh otimo
44c 		pego[id] = true;
3fd 		recuperar(id+1, cap-w[id]);
4ee 	} else { // nao pegar eh otimo
884 		pego[id] = false;
45d 		recuperar(id+1, cap);
549 	}
845 }


63d void solve() {
    
311 	int cap; cin >> n >> cap;
457 	memset(memo, -1, sizeof memo);
    
03b 	f(i,0,n) { cin >> w[i] >> v[i]; }
    
304 	int lucro_max = dp(0, cap);
    
ae7 	recuperar(0, cap);
    	
4cc 	int lucro = 0, peso = 0;
418 	f(i,0,n) {
ecd 		if(pego[i]) {
73f 			lucro += v[i];
20e 			peso += w[i];
c3f 		}
b7f 	}
    	
d13 	assert(lucro_max == lucro and peso <= cap);
f6f }
\end{lstlisting}



%%%%%%%%%%%%%%%%%%%%
%
% Estruturas
%
%%%%%%%%%%%%%%%%%%%%

\section{Estruturas}

\subsection{BIT}
\begin{lstlisting}
// BIT de soma 0-based
// 
// upper_bound(x) retorna o menor p tal que pref(p) > x 
//
// Complexidades:
// build - O(n)
// update - O(log(n))
// query - O(log(n))
// upper_bound - O(log(n))

8eb struct Bit {
1a8 	int n;
116 	vector<int> bit;
e86 	Bit(int _n=0) : n(_n), bit(n + 1) {}
70f 	Bit(vector<int>& v) : n(v.size()), bit(n + 1) {
78a 		for (int i = 1; i <= n; i++) {
671 			bit[i] += v[i - 1];
edf 			int j = i + (i & -i);
b8a 			if (j <= n) bit[j] += bit[i];
806 		}
e89 	}
cf6 	void update(int i, int x) { // soma x na posicao i
b64 		for (i++; i <= n; i += i & -i) bit[i] += x;
850 	}
cdb 	int pref(int i) { // soma [0, i]
7c9 		int ret = 0;
4d3 		for (i++; i; i -= i & -i) ret += bit[i];
edf 		return ret;
065 	}
9e3 	int query(int l, int r) {  // soma [l, r]
89b 		return pref(r) - pref(l - 1); 
e97 	}
bdf 	int upper_bound(int x) {
1ba 		int p = 0;
0af 		for (int i = __lg(n); i+1; i--) 
6f5 			if (p + (1<<i) <= n and bit[p + (1<<i)] <= x)
68e 				x -= bit[p += (1 << i)];
74e 		return p;
be4 	}
f75 };

63d void solve() {
    
70a 	vector<int> v(n);
    
8b8 	Bit bit (v);
edc }
\end{lstlisting}

\subsection{DSU}
\begin{lstlisting}
// Une dois conjuntos e acha a qual conjunto um elemento pertence por seu id
//
// find e unite: O(a(n)) ~= O(1) amortizado

8d3 struct dsu {
825 	vector<int> id, sz;
    
b33 	dsu(int n) : id(n), sz(n, 1) { iota(id.begin(), id.end(), 0); }
    
0cf 	int find(int a) { return a == id[a] ? a : id[a] = find(id[a]); }
    
440 	void unite(int a, int b) {
605 		a = find(a), b = find(b);
d54 		if (a == b) return;
956 		if (sz[a] < sz[b]) swap(a, b);
6d0 		sz[a] += sz[b], id[b] = a;
ea7 	}
8e1 };

// DSU de bipartido
//
// Une dois vertices e acha a qual componente um vertice pertence
// Informa se a componente de um vertice e bipartida
//
// find e unite: O(log(n))

8d3 struct dsu {
6f7 	vector<int> id, sz, bip, c;
    
5b4 	dsu(int n) : id(n), sz(n, 1), bip(n, 1), c(n) { 
db8 		iota(id.begin(), id.end(), 0); 
f25 	}
    
ef0 	int find(int a) { return a == id[a] ? a : find(id[a]); }
f30 	int color(int a) { return a == id[a] ? c[a] : c[a] ^ color(id[a]); }
    
440 	void unite(int a, int b) {
263 		bool change = color(a) == color(b);
605 		a = find(a), b = find(b);
a89 		if (a == b) {
4ed 			if (change) bip[a] = 0;
505 			return;
32d 		}
    		
956 		if (sz[a] < sz[b]) swap(a, b);
efe 		if (change) c[b] = 1;
2cd 		sz[a] += sz[b], id[b] = a, bip[a] &= bip[b];
22b 	}
118 };


// DSU Persistente
//
// Persistencia parcial, ou seja, tem que ir
// incrementando o 't' no une
//
// find e unite: O(log(n))

8d3 struct dsu {
33c 	vector<int> id, sz, ti;
    
733 	dsu(int n) : id(n), sz(n, 1), ti(n, -INF) { 
db8 		iota(id.begin(), id.end(), 0); 
aad 	}
    
5e6 	int find(int a, int t) { 
6ba 		if (id[a] == a or ti[a] > t) return a;
ea5 		return find(id[a], t);
6cb 	}
    
fa0 	void unite(int a, int b, int t) {
84f 		a = find(a, t), b = find(b, t);
d54 		if (a == b) return;
956 		if (sz[a] < sz[b]) swap(a, b);
35d 		sz[a] += sz[b], id[b] = a, ti[b] = t;
513 	}
6c6 };

// DSU com rollback
//
// checkpoint(): salva o estado atual de todas as variaveis
// rollback(): retorna para o valor das variaveis para 
// o ultimo checkpoint
//
// Sempre que uma variavel muda de valor, adiciona na stack
//
// find e unite: O(log(n))
// checkpoint: O(1)
// rollback: O(m) em que m e o numero de vezes que alguma
// variavel mudou de valor desde o ultimo checkpoint

8d3 struct dsu {
825 	vector<int> id, sz;
27c 	stack<stack<pair<int&, int>>> st;
    
98d 	dsu(int n) : id(n), sz(n, 1) { 
1cc 		iota(id.begin(), id.end(), 0), st.emplace(); 
8cd 	}
    		
bdf 	void save(int &x) { st.top().emplace(x, x); }
    
30d 	void checkpoint() { st.emplace(); }
    	
5cf 	void rollback() {
ba9 		while(st.top().size()) {
6bf 			auto [end, val] = st.top().top(); st.top().pop();
149 			end = val;
f9a 		}
25a 		st.pop();
3c6 	}
    
ef0 	int find(int a) { return a == id[a] ? a : find(id[a]); }
    
440 	void unite(int a, int b) {
605 		a = find(a), b = find(b);
d54 		if (a == b) return;
956 		if (sz[a] < sz[b]) swap(a, b);
803 		save(sz[a]), save(id[b]);
6d0 		sz[a] += sz[b], id[b] = a;
1b9 	}
c6e };
\end{lstlisting}

\subsection{Fenwick Tree (BIT) Range }
\begin{lstlisting}
// Operacoes 0-based
// query(l, r) retorna a soma de v[l..r]
// update(l, r, x) soma x em v[l..r]
//
// Complexidades:
// build - O(n)
// query - O(log(n))
// update - O(log(n))

796 namespace BitRange {
06d 	int bit[2][MAX+2];
1a8 	int n;
    
727 	void build(int n2, vector<int>& v) {
1e3 		n = n2;
535 		for (int i = 1; i <= n; i++)
a6e 			bit[1][min(n+1, i+(i&-i))] += bit[1][i] += v[i];
d31 	}
1a7 	int get(int x, int i) {
7c9 		int ret = 0;
360 		for (; i; i -= i&-i) ret += bit[x][i];
edf 		return ret;
a4e 	}
920 	void add(int x, int i, int val) {
503 		for (; i <= n; i += i&-i) bit[x][i] += val;
fae 	}
3d9 	int get2(int p) {
c7c 		return get(0, p) * p + get(1, p);
33c 	}
9e3 	int query(int l, int r) { // zero-based
ff5 		return get2(r+1) - get2(l);
25e 	}
7ff 	void update(int l, int r, int x) {
e5f 		add(0, l+1, x), add(0, r+2, -x);
f58 		add(1, l+1, -x*l), add(1, r+2, x*(r+1));
5ce 	}
1b4 };

63d void solve() {
    
97a 	vector<int> v {0,1,2,3,4,5}; // v[0] eh inutilizada
f98 	BitRange::build(v.size(), v);
    
67f 	int a = 0, b = 3;
3d5 	BitRange::query(a, b); // v[a] + v[a+1] + ... + v[b] = 6  | 1+2+3 = 6 | zero-based
a4b 	BitRange::update(a, b, 2); // v[a...b] += 2 | zero-based
d65 }
\end{lstlisting}

\subsection{SegTree}
\begin{lstlisting}
// Recursiva com Lazy Propagation
// Query: soma do range [a, b]
// Update: soma x em cada elemento do range [a, b]
// Pode usar a seguinte funcao para indexar os nohs:
// f(l, r) = (l+r)|(l!=r), usando 2N de memoria
//
// Complexidades:
// build - O(n)
// query - O(log(n))
// update - O(log(n))

0d2 const int MAX = 1e5+10;

fb1 namespace SegTree {
098 	int seg[4*MAX], lazy[4*MAX];
052 	int n, *v;
    
b90 	int op(int a, int b) { return a + b; }
    
2c4 	int build(int p=1, int l=0, int r=n-1) {
3c7 		lazy[p] = 0;
6cd 		if (l == r) return seg[p] = v[l];
ee4 		int m = (l+r)/2;
317 		return seg[p] = op(build(2*p, l, m), build(2*p+1, m+1, r));
985 	}
    
0d8 	void build(int n2, int* v2) {
680 		n = n2, v = v2;
6f2 		build();
acb 	}
    
ceb 	void prop(int p, int l, int r) {
cdf 		seg[p] += lazy[p]*(r-l+1);
2c9 		if (l != r) lazy[2*p] += lazy[p], lazy[2*p+1] += lazy[p];
3c7 		lazy[p] = 0;
c10 	}
    
04a 	int query(int a, int b, int p=1, int l=0, int r=n-1) {
6b9 		prop(p, l, r);
527 		if (a <= l and r <= b) return seg[p];
786 		if (b < l or r < a) return 0;
ee4 		int m = (l+r)/2;
19e 		return op(query(a, b, 2*p, l, m), query(a, b, 2*p+1, m+1, r));
1c9 	}
    	
f33 	int update(int a, int b, int x, int p=1, int l=0, int r=n-1) {
6b9 		prop(p, l, r);
9a3 		if (a <= l and r <= b) {
b94 			lazy[p] += x;
6b9 			prop(p, l, r);
534 			return seg[p];
821 		}
e9f 		if (b < l or r < a) return seg[p];
ee4 		int m = (l+r)/2;
a8f 		return seg[p] = op(update(a, b, x, 2*p, l, m), update(a, b, x, 2*p+1, m+1, r));
08f 	}
    
    	// Se tiver uma seg de max, da pra descobrir em O(log(n))
    	// o primeiro e ultimo elemento >= val numa range:
    
    	// primeira posicao >= val em [a, b] (ou -1 se nao tem)
119 	int get_left(int a, int b, int val, int p=1, int l=0, int r=n-1) {
6b9 		prop(p, l, r);
f38 		if (b < l or r < a or seg[p] < val) return -1;
205 		if (r == l) return l;
ee4 		int m = (l+r)/2;
753 		int x = get_left(a, b, val, 2*p, l, m);
50e 		if (x != -1) return x;
c3c 		return get_left(a, b, val, 2*p+1, m+1, r);
68c 	}
    
    	// ultima posicao >= val em [a, b] (ou -1 se nao tem)
992 	int get_right(int a, int b, int val, int p=1, int l=0, int r=n-1) {
6b9 		prop(p, l, r);
f38 		if (b < l or r < a or seg[p] < val) return -1;
205 		if (r == l) return l;
ee4 		int m = (l+r)/2;
1b1 		int x = get_right(a, b, val, 2*p+1, m+1, r);
50e 		if (x != -1) return x;
6a7 		return get_right(a, b, val, 2*p, l, m);
1b7 	}
    
    	// Se tiver uma seg de soma sobre um array nao negativo v, da pra
    	// descobrir em O(log(n)) o maior j tal que v[i]+v[i+1]+...+v[j-1] < val
89b 	int lower_bound(int i, int& val, int p, int l, int r) {
6b9 		prop(p, l, r);
6e8 		if (r < i) return n;
b5d 		if (i <= l and seg[p] < val) {
bff 			val -= seg[p];
041 			return n;
634 		}
3ce 		if (l == r) return l;
ee4 		int m = (l+r)/2;
514 		int x = lower_bound(i, val, 2*p, l, m);
ee0 		if (x != n) return x;
8b9 		return lower_bound(i, val, 2*p+1, m+1, r);
01d 	}
a15 };

63d void solve() {
213 	int n = 10;
89e 	int v[] = {1, 2, 3, 4, 5, 6, 7, 8, 9, 10};
2d5 	SegTree::build(n, v);
    
3af 	cout << SegTree::query(0, 9) << endl; // seg[0] + seg[1] + ... + seg[9] = 55
310 	SegTree::update(0, 9, 1); // seg[0,...,9] += 1
6d9 }
\end{lstlisting}

\subsection{Sparse Table Disjunta}
\begin{lstlisting}
// Description: Sparse Table Disjunta para soma de intervalos
// Complexity Temporal: O(n log n) para construir e O(1) para consultar
// Complexidade Espacial: O(n log n)

125 const int MAX = 100010
d5a const int MAX2 = 20 // log(MAX)

82d namespace SparseTable {
9bf 	int m[MAX2][2*MAX], n, v[2*MAX];
b90 	int op(int a, int b) { return a + b; }
0d8 	void build(int n2, int* v2) {
1e3 		n = n2;
df4 		for (int i = 0; i < n; i++) v[i] = v2[i];
a84 		while (n&(n-1)) n++;
3d2 		for (int j = 0; (1<<j) < n; j++) {
1c0 			int len = 1<<j;
d9b 			for (int c = len; c < n; c += 2*len) {
332 				m[j][c] = v[c], m[j][c-1] = v[c-1];
668 				for (int i = c+1; i <  c+len; i++) m[j][i] = op(m[j][i-1], v[i]);
432 				for (int i = c-2; i >= c-len; i--) m[j][i] = op(v[i], m[j][i+1]);
eda 			}
f4d 		}
ce3 	}
9e3 	int query(int l, int r) {
f13 		if (l == r) return v[l];
e6d 		int j = __builtin_clz(1) - __builtin_clz(l^r);
d67 		return op(m[j][l], m[j][r]);
a7b 	}
258 }

63d void solve() {
ce1 	int n = 9;
1a3 	int v[] = {1, 2, 3, 4, 5, 6, 7, 8, 9};
3f7 	SparseTable::build(n, v);
925 	cout << SparseTable::query(0, n-1) << endl; // sparse[0] + sparse[1] + ... + sparse[n-1] = 45
241 }
\end{lstlisting}

\subsection{Tabuleiro}
\begin{lstlisting}
// Description: Estrutura que simula um tabuleiro M x N, sem realmente criar uma matriz
// Permite atribuir valores a linhas e colunas, e consultar a posicao mais frequente
// Complexidade Atribuir: O(log(N))
// Complexidade Consulta: O(log(N))
// Complexidade verificar frequencia geral: O(N * log(N))
9a0 #define MAX_VAL 5 // maior valor que pode ser adicionado na matriz + 1

8ee class BinTree {
d9d     protected:
ef9         vector<int> mBin;
673     public:
d5e         explicit BinTree(int n) { mBin = vector(n + 1, 0); }
    
e44         void add(int p, const int val) {
dd1             for (auto size = mBin.size(); p < size; p += p & -p)
174                 mBin[p] += val;
b68         }
    
e6b         int query(int p) {
e1c             int sumToP {0};
b62             for (; p > 0; p -= p & -p)
ec1                 sumToP += mBin[p];
838             return sumToP;
793         }
a5f };

b6a class ReverseBinTree : public BinTree {
673     public:
83e         explicit ReverseBinTree(int n) : BinTree(n) {};
    
e44         void add(int p, const int val) {
850             BinTree::add(static_cast<int>(mBin.size()) - p, val);
705         }
    
e6b         int query(int p) {
164             return BinTree::query(static_cast<int>(mBin.size()) - p);
a21         }
6cf };

952 class Tabuleiro {
673     public:
177         explicit Tabuleiro(const int m, const int n, const int q) : mM(m), mN(n), mQ(q) {
958             mLinhas = vector<pair<int, int8_t>>(m, {0, 0});
d68             mColunas = vector<pair<int, int8_t>>(n, {0, 0});
    
66e             mAtribuicoesLinhas = vector(MAX_VAL, ReverseBinTree(mQ)); // aARvore[51]
9e5             mAtribuicoesColunas = vector(MAX_VAL, ReverseBinTree(mQ));
13b         }
    
bc2         void atribuirLinha(const int x, const int8_t r) {
e88             mAtribuirFileira(x, r, mLinhas, mAtribuicoesLinhas);
062         }
    
ca2         void atribuirColuna(const int x, const int8_t r) {
689             mAtribuirFileira(x, r, mColunas, mAtribuicoesColunas);
a40         }
    
d10         int maxPosLinha(const int x) {
f95             return mMaxPosFileira(x, mLinhas, mAtribuicoesColunas, mM);
8ba         }
    
ff7         int maxPosColuna(const int x) {
b95             return mMaxPosFileira(x, mColunas, mAtribuicoesLinhas, mN);
252         }
    
80e         vector<int> frequenciaElementos() {
    
a35             vector<int> frequenciaGlobal(MAX_VAL, 0);
45a             for(int i=0; i<mM; i++) {
ebd                 vector<int> curr = frequenciaElementos(i, mAtribuicoesColunas);
97f                 for(int j=0; j<MAX_VAL; j++)
ef3                     frequenciaGlobal[j] += curr[j];
094             }
01e             return frequenciaGlobal;
b7a         }
    
bf2     private:
69d         int mM, mN, mQ, mMoment {0};
    
0a6         vector<ReverseBinTree> mAtribuicoesLinhas, mAtribuicoesColunas;
f2d         vector<pair<int, int8_t>> mLinhas, mColunas;
    
e7a         void mAtribuirFileira(const int x, const int8_t r, vector<pair<int, int8_t>>& fileiras,
1d7                             vector<ReverseBinTree>& atribuicoes) {
224             if (auto& [oldQ, oldR] = fileiras[x]; oldQ)
bda                 atribuicoes[oldR].add(oldQ, -1);
    
914             const int currentMoment = ++mMoment;
b2c             fileiras[x].first = currentMoment;
80b             fileiras[x].second = r;
f65             atribuicoes[r].add(currentMoment, 1);
5de         }
    
2b8         int mMaxPosFileira(const int x, const vector<pair<int, int8_t>>& fileiras, vector<ReverseBinTree>& atribuicoesPerpendiculares, const int& currM) const {
1aa             auto [momentoAtribuicaoFileira, rFileira] = fileiras[x];
    
8d0             vector<int> fileiraFrequencia(MAX_VAL, 0);
729             fileiraFrequencia[rFileira] = currM;
    
85a             for (int8_t r {0}; r < MAX_VAL; ++r) {
8ca                 const int frequenciaR = atribuicoesPerpendiculares[r].query(momentoAtribuicaoFileira + 1);
04a                 fileiraFrequencia[rFileira] -= frequenciaR;
72e                 fileiraFrequencia[r] += frequenciaR;
6b0             }
    
b59             return MAX_VAL - 1 - (max_element(fileiraFrequencia.crbegin(), fileiraFrequencia.crend()) - fileiraFrequencia.crbegin());
372         }
    
7c4         vector<int> frequenciaElementos(int x, vector<ReverseBinTree>& atribuicoesPerpendiculares) const {
                
8d0             vector<int> fileiraFrequencia(MAX_VAL, 0);
    
583             auto [momentoAtribuicaoFileira, rFileira] = mLinhas[x];
    
083             fileiraFrequencia[rFileira] = mN;
    
85a             for (int8_t r {0}; r < MAX_VAL; ++r) {
8ca                 const int frequenciaR = atribuicoesPerpendiculares[r].query(momentoAtribuicaoFileira + 1);
04a                 fileiraFrequencia[rFileira] -= frequenciaR;
72e                 fileiraFrequencia[r] += frequenciaR;
6b0             }
    
2e6             return fileiraFrequencia;
15d         }
    
20c };

63d void solve() {
    
e29     int L, C, q; cin >> L >> C >> q;
    
56c     Tabuleiro tabuleiro(L, C, q);
    
a09     int linha = 0, coluna = 0, valor = 10; // linha e coluna sao 0 based
b68     tabuleiro.atribuirLinha(linha, static_cast<int8_t>(valor)); // f(i,0,C) matriz[linha][i] = valor
34d     tabuleiro.atribuirColuna(coluna, static_cast<int8_t>(valor)); // f(i,0,L) matriz[i][coluna] = valor
    
        // Freuencia de todos os elementos, de 0 a MAX_VAL-1
155     vector<int> frequenciaGeral = tabuleiro.frequenciaElementos();
    
176     int a = tabuleiro.maxPosLinha(linha); // retorna a posicao do elemento mais frequente na linha
981     int b = tabuleiro.maxPosColuna(coluna); // retorna a posicao do elemento mais frequente na coluna
9b5 }
\end{lstlisting}

\subsection{Union-Find (Disjoint Set Union)}
\begin{lstlisting}
da5 const int MAXN = 1e3+10;

0cd struct UnionFind {
c55     int numSets;
320     int id[MAXN], sz[MAXN];
    
02d     UnionFind(int N) {
0bd         numSets = N;
faa         for (int i = 0; i < N; i++) {
963             id[i] = i;
02c             sz[i] = 1;
0e9         }
41f     }
    
aee     int find(int a) { 
3da         return id[a] = (id[a] == a ? a : find(id[a]));
c7b     }
    
78b     void uni(int a, int b) { 
605         a = find(a), b = find(b);
d54         if(a == b) return;
3c6         if(sz[a] > sz[b]) swap(a, b);
351         id[a] = b;
582         sz[b] += sz[a];
92a         --numSets;
650     }
        
3ae     int sizeOfSet(int a) { return sz[find(a)]; }
c59     int numDisjointSets() { return numSets; }
864 };


63d void solve() {
    
f98 	int n, ed; cin >> n >> ed;
f4e 	UnionFind uni(n);
    
31c 	f(i,0,ed) {
602 		int a, b; cin >> a >> b; a--, b--;
45e 		uni.uni(a,b);
c0f 	}
    
350 	cout << uni.numDisjointSets() << endl;
01b }
\end{lstlisting}



%%%%%%%%%%%%%%%%%%%%
%
% Grafos
%
%%%%%%%%%%%%%%%%%%%%

\section{Grafos}

\subsection{APSP - Floyd Warshall}
\begin{lstlisting}
// Calcula caminho minimo entre todos os pares de vertices
// O(V^3)

b94 const int MAXN = 110;

0eb int adj[MAXN][MAXN];

eac void printAnswer(int n) {
5bf 	for (int u = 0; u < n; ++u)
a3f 	for (int v = 0; v < n; ++v)
6ab 	cout << "APSP("<<u<<", "<<v<<") = " << adj[u][v] << endl;
8c7 }

e4a void prepareParent() {
418 	f(i,0,n) {
001 		f(j,0,n) {
b5a 			p[i][j] = i;
441 		}
d89 	}
    
b9b 	for (int k = 0; k < n; ++k)
6cb 	for (int i = 0; i < n; ++i)
a9e 	for (int j = 0; j < n; ++j)
c1d 		if (adj[i][k] + adj[k][j] < adj[i][j]) {
a1a 			adj[i][j] = adj[i][k]+adj[k][j];
9b6 			p[i][j] = p[k][j];
a04 		}
c85 }

470 vector<int> restorePath(int u, int v) {
    
36d 	if (adj[u][v] == INF) return {};
5b4 	vector<int>  path;
81f 	for (; v != u; v = p[u][v]) {
ff8 		if (v == -1) return {};
80a 		path.push_back(v);
3d1 	}
960 	path.push_back(u);
3a9 	reverse(path.begin(), path.end());
535 	return path;
16a }

230 void floyd_warshall(int n) {
e22 	for (int k = 0; k < n; k++)
830 	for (int i = 0; i < n; i++)
f90 	for (int j = 0; j < n; j++)
ffd 		adj[i][j] = min(adj[i][j], adj[i][k] + adj[k][j]);
e78 }

e03 void solve(int n, int ed) {
    
418 	f(i,0,n) {
001 		f(j,0,n) {
59d 			adj[i][j] = INF;
93d 		}
774 		adj[i][i] = 0;
c4e 	}
    
c92 	while(ed--) {
c48 		int u, v, w; cin >> u >> v >> w; u--, v--;
7da 		adj[u][v] = adj[v][u] = w;
2a0 	}
    
803 	floyd_warshall(n);
    
    	// prepareParent();
    	// auto path = restorePath(0, 3);
649 }

// Diametro do Grafo: maior valor de adj[u][v] != INF
\end{lstlisting}

\subsection{BFS}
\begin{lstlisting}
// O(V + E)

2a7 const int MAXN = 1e4+10;

465 bool vis[MAXN];
be4 int d[MAXN], p[MAXN];
ea6 vector<int> adj [MAXN];

94c void bfs(int s) {
    
8b2     queue<int> q; q.push(s);
654     vis[s] = true, d[s] = 0, p[s] = -1;
    
14d     while (!q.empty()) {
0e6         int v = q.front(); q.pop(); 
c25         vis[v] = true;
            
f74         for (int u : adj[v]) {
1d6             if (!vis[u]) {
b9c                 vis[u] = true;
f73                 q.push(u);
                    // d[u] = d[v] + 1;
                    // p[u] = v;
3b2             }
3d2         }
cc2     }
75a }

63d void solve() {
98f     int n, ed; 
        
19e 	f(i,0,n) { d[i] = -1, p[i] = -1; }    
    	
c92 	while(ed--) {
ed2         int u, v; cin >> u >> v; u--, v--;
cc9         adj[u].push_back(v);
1ea         adj[v].push_back(u);
3f1     }
    
19c 	int s; bfs(s);
81a }

// Checar Fortemente Conexo: BFS adj e adj reverso, ver se todos os vertices foram visitados.
\end{lstlisting}

\subsection{BFS - por niveis}
\begin{lstlisting}
//  Encontrar distancia entre S e outros pontos em que pontos estao agrupados (terminais)

ef8 const int MAXN = 510;
57d const int MAXEDG = 510; // maximo numero de terminais

9d3 int dist[MAXN];
a19 vector<int> niveisDoNode[MAXN], nodesDoNivel[MAXEDG];

94c void bfs(int s) {
    
735     queue<pair<int, int>> q; q.push({s, 0});
    
a93 	dist[s] = 0;
    
14d     while (!q.empty()) {
2bc         auto [v, dis] = q.front(); q.pop(); 
    
400         for(auto nivel : niveisDoNode[v]) {
8fd             for(auto u : nodesDoNivel[nivel]) {
619                 if (dist[u] == 0) {
324                     q.push({u, dis+1});
554                     dist[u] = dis + 1;
12f                 }
46b             }
ffe         }
e19     }
e00 }

63d void solve() {
    
09d 	int n, terminais, s, e;
    
6bf 	f(i,0,terminais) {
509 		int q; cin >> q;
1f4 		while(q--) {
9aa 			int v; v--;
e19 			niveisDoNode[v].push_back(i);
b14 			nodesDoNivel[i].push_back(v);
fd8 		}
6ec 	}
    
aff 	bfs(s);
    
85a 	cout << dist[e] << endl;
7b3 }
\end{lstlisting}

\subsection{Emparelhamento Max Grafo Bipartido (Kuhn)}
\begin{lstlisting}
// Computa matching maximo em grafo bipartido
// 'n' e 'm' sao quantos vertices tem em cada particao
// chamar add(i, j) para add aresta entre o cara i
// da particao A, e o cara j da particao B
// (entao i < n, j < m)
// Para recuperar o matching, basta olhar 'ma' e 'mb'
// 'recover' recupera o min vertex cover como um par de
// {caras da particao A, caras da particao B}
//
// O(|V| * |E|)
// Na pratica, parece rodar tao rapido quanto o Dinitz


878 mt19937 rng((int) chrono::steady_clock::now().time_since_epoch().count());

6c6 struct kuhn {
14e 	int n, m;
789 	vector<vector<int>> g;
d3f 	vector<int> vis, ma, mb;
    
40e 	kuhn(int n_, int m_) : n(n_), m(m_), g(n),
8af 		vis(n+m), ma(n, -1), mb(m, -1) {}
    
ba6 	void add(int a, int b) { g[a].push_back(b); }
    
caf 	bool dfs(int i) {
29a 		vis[i] = 1;
29b 		for (int j : g[i]) if (!vis[n+j]) {
8c9 			vis[n+j] = 1;
2cf 			if (mb[j] == -1 or dfs(mb[j])) {
bfe 				ma[i] = j, mb[j] = i;
8a6 				return true;
b17 			}
82a 		}
d1f 		return false;
4ef 	}
bf7 	int matching() {
1ae 		int ret = 0, aum = 1;
5a8 		for (auto& i : g) shuffle(i.begin(), i.end(), rng);
392 		while (aum) {
618 			for (int j = 0; j < m; j++) vis[n+j] = 0;
c5d 			aum = 0;
830 			for (int i = 0; i < n; i++)
01f 				if (ma[i] == -1 and dfs(i)) ret++, aum = 1;
085 		}
edf 		return ret;
2ee 	}
b0d };

63d void solve() {
    
be0 	int n1; // Num vertices lado esquerdo grafo bipartido
e4c 	int n2; // Num vertices lado direito grafo bipartido
    	
761 	kuhn K(n1, n2);
    
732 	int edges;
    
6e0 	while(edges--) {
b1f 		int a, b; cin >> a >> b;
3dc 		K.add(a,b); // a -> b
5b7 	}
    
69b 	int emparelhamentoMaximo = K.matching();
76b }
\end{lstlisting}

\subsection{Fluxo - Dinitz (Max Flow)}
\begin{lstlisting}
// Encontra fluxo maximo de um grafo
// O(min(m * max_flow, n^2 m))
// Grafo com capacidades 1: O(min(m sqrt(m), m * n^(2/3)))
// Todo vertice tem grau de entrada ou saida 1: O(m sqrt(n))

472 struct dinitz {
61f 	const bool scaling = false; // com scaling -> O(nm log(MAXCAP)),
206 	int lim;                    // com constante alta
670 	struct edge {
358 		int to, cap, rev, flow;
7f9 		bool res;
d36 		edge(int to_, int cap_, int rev_, bool res_)
a94 			: to(to_), cap(cap_), rev(rev_), flow(0), res(res_) {}
f70 	};
    
002 	vector<vector<edge>> g;
216 	vector<int> lev, beg;
a71 	ll F;
190 	dinitz(int n) : g(n), F(0) {}
    
087 	void add(int a, int b, int c) {
bae 		g[a].emplace_back(b, c, g[b].size(), false);
4c6 		g[b].emplace_back(a, 0, g[a].size()-1, true);
5c2 	}
123 	bool bfs(int s, int t) {
90f 		lev = vector<int>(g.size(), -1); lev[s] = 0;
64c 		beg = vector<int>(g.size(), 0);
8b2 		queue<int> q; q.push(s);
402 		while (q.size()) {
be1 			int u = q.front(); q.pop();
bd9 			for (auto& i : g[u]) {
dbc 				if (lev[i.to] != -1 or (i.flow == i.cap)) continue;
b4f 				if (scaling and i.cap - i.flow < lim) continue;
185 				lev[i.to] = lev[u] + 1;
8ca 				q.push(i.to);
f97 			}
e87 		}
0de 		return lev[t] != -1;
742 	}
dfb 	int dfs(int v, int s, int f = INF) {
50b 		if (!f or v == s) return f;
88f 		for (int& i = beg[v]; i < g[v].size(); i++) {
027 			auto& e = g[v][i];
206 			if (lev[e.to] != lev[v] + 1) continue;
ee0 			int foi = dfs(e.to, s, min(f, e.cap - e.flow));
749 			if (!foi) continue;
3c5 			e.flow += foi, g[e.to][e.rev].flow -= foi;
45c 			return foi;
618 		}
bb3 		return 0;
4b1 	}
ff6 	ll max_flow(int s, int t) {
a86 		for (lim = scaling ? (1<<30) : 1; lim; lim /= 2)
9d1 			while (bfs(s, t)) while (int ff = dfs(s, t)) F += ff;
4ff 		return F;
8b9 	}
86f };

// Recupera as arestas do corte s-t
dbd vector<pair<int, int>> get_cut(dinitz& g, int s, int t) {
f07 	g.max_flow(s, t);
68c 	vector<pair<int, int>> cut;
1b0 	vector<int> vis(g.g.size(), 0), st = {s};
321 	vis[s] = 1;
3c6 	while (st.size()) {
b17 		int u = st.back(); st.pop_back();
322 		for (auto e : g.g[u]) if (!vis[e.to] and e.flow < e.cap)
c17 			vis[e.to] = 1, st.push_back(e.to);
d14 	}
481 	for (int i = 0; i < g.g.size(); i++) for (auto e : g.g[i])
9d2 		if (vis[i] and !vis[e.to] and !e.res) cut.emplace_back(i, e.to);
d1b 	return cut;
1e8 }

63d void solve() {
    
1a8 	int n; // numero de arestas
b06 	dinitz g(n);
    
732 	int edges;
6e0 	while(edges--) {
1e1 		int a, b, w; cin >> a >> b >> c;
f93 		g.add(a,b,c); // a -> b com capacidade c
fa1 	}
    
07a 	int maxFlow = g.max_flow(SRC, SNK); // max flow de SRC -> SNK
a7b }
\end{lstlisting}

\subsection{Fluxo - MinCostMaxFlow}
\begin{lstlisting}
// min_cost_flow(s, t, f) computa o par (fluxo, custo)
// com max(fluxo) <= f que tenha min(custo)
// min_cost_flow(s, t) -> Fluxo maximo de custo minimo de s pra t
// Se for um dag, da pra substituir o SPFA por uma DP pra nao
// pagar O(nm) no comeco
// Se nao tiver aresta com custo negativo, nao precisa do SPFA
//
// O(nm + f * m log n)

123 template<typename T> struct mcmf {
670 	struct edge {
b75 		int to, rev, flow, cap; // para, id da reversa, fluxo, capacidade
7f9 		bool res; // se eh reversa
635 		T cost; // custo da unidade de fluxo
892 		edge() : to(0), rev(0), flow(0), cap(0), cost(0), res(false) {}
1d7 		edge(int to_, int rev_, int flow_, int cap_, T cost_, bool res_)
f8d 			: to(to_), rev(rev_), flow(flow_), cap(cap_), res(res_), cost(cost_) {}
723 	};
    
002 	vector<vector<edge>> g;
168 	vector<int> par_idx, par;
f1e 	T inf;
a03 	vector<T> dist;
    
b22 	mcmf(int n) : g(n), par_idx(n), par(n), inf(numeric_limits<T>::max()/3) {}
    
91c 	void add(int u, int v, int w, T cost) { // de u pra v com cap w e custo cost
2fc 		edge a = edge(v, g[v].size(), 0, w, cost, false);
234 		edge b = edge(u, g[u].size(), 0, 0, -cost, true);
    
b24 		g[u].push_back(a);
c12 		g[v].push_back(b);
0ed 	}
    
8bc 	vector<T> spfa(int s) { // nao precisa se nao tiver custo negativo
871 		deque<int> q;
3d1 		vector<bool> is_inside(g.size(), 0);
577 		dist = vector<T>(g.size(), inf);
    
a93 		dist[s] = 0;
a30 		q.push_back(s);
ecb 		is_inside[s] = true;
    
14d 		while (!q.empty()) {
b1e 			int v = q.front();
ced 			q.pop_front();
48d 			is_inside[v] = false;
    
76e 			for (int i = 0; i < g[v].size(); i++) {
9d4 				auto [to, rev, flow, cap, res, cost] = g[v][i];
e61 				if (flow < cap and dist[v] + cost < dist[to]) {
943 					dist[to] = dist[v] + cost;
    
ed6 					if (is_inside[to]) continue;
020 					if (!q.empty() and dist[to] > dist[q.front()]) q.push_back(to);
b33 					else q.push_front(to);
b52 					is_inside[to] = true;
2d1 				}
8cd 			}
f2c 		}
8d7 		return dist;
96c 	}
2a2 	bool dijkstra(int s, int t, vector<T>& pot) {
489 		priority_queue<pair<T, int>, vector<pair<T, int>>, greater<>> q;
577 		dist = vector<T>(g.size(), inf);
a93 		dist[s] = 0;
115 		q.emplace(0, s);
402 		while (q.size()) {
91b 			auto [d, v] = q.top();
833 			q.pop();
68b 			if (dist[v] < d) continue;
76e 			for (int i = 0; i < g[v].size(); i++) {
9d4 				auto [to, rev, flow, cap, res, cost] = g[v][i];
e8c 				cost += pot[v] - pot[to];
e61 				if (flow < cap and dist[v] + cost < dist[to]) {
943 					dist[to] = dist[v] + cost;
441 					q.emplace(dist[to], to);
88b 					par_idx[to] = i, par[to] = v;
873 				}
de3 			}
9d4 		}
1d4 		return dist[t] < inf;
c68 	}
    
3d2 	pair<int, T> min_cost_flow(int s, int t, int flow = INF) {
3dd 		vector<T> pot(g.size(), 0);
9e4 		pot = spfa(s); // mudar algoritmo de caminho minimo aqui
    
d22 		int f = 0;
ce8 		T ret = 0;
4a0 		while (f < flow and dijkstra(s, t, pot)) {
bda 			for (int i = 0; i < g.size(); i++)
d2a 				if (dist[i] < inf) pot[i] += dist[i];
    
71b 			int mn_flow = flow - f, u = t;
045 			while (u != s){
90f 				mn_flow = min(mn_flow,
07d 					g[par[u]][par_idx[u]].cap - g[par[u]][par_idx[u]].flow);
3d1 				u = par[u];
935 			}
    
1f2 			ret += pot[t] * mn_flow;
    
476 			u = t;
045 			while (u != s) {
e09 				g[par[u]][par_idx[u]].flow += mn_flow;
d98 				g[u][g[par[u]][par_idx[u]].rev].flow -= mn_flow;
3d1 				u = par[u];
bcc 			}
    
04d 			f += mn_flow;
36d 		}
    
15b 		return make_pair(f, ret);
cc3 	}
    
    	// Opcional: retorna as arestas originais por onde passa flow = cap
182 	vector<pair<int,int>> recover() {
24a 		vector<pair<int,int>> used;
2a4 		for (int i = 0; i < g.size(); i++) for (edge e : g[i])
587 			if(e.flow == e.cap && !e.res) used.push_back({i, e.to});
f6b 		return used;
390 	}
697 };

63d void solve(){
    
1a8 	int n; // numero de vertices
4c5 	mcmf<int> mincost(n);
    
ab4 	mincost.add(u, v, cap, cost); // unidirecional
983 	mincost.add(v, u, cap, cost); // bidirecional
    
073 	auto [flow, cost] = mincost.min_cost_flow(src, end/*, initialFlow*/);
    
da5 }
\end{lstlisting}

\subsection{Fluxo - Problemas}
\begin{lstlisting}
// 1: Problema do Corte
7a9 - Entrada:
bc1 	- N itens
388 	- Curso Wi Inteiro
7c3 	- M restricoes: se eu pegar Ai, eu preciso pegar Bi...
387 - Saida: valor maximo pegavel

ac2 - Solucao: corte maximo com Dinitz
019 	- dinitz(n+m+1)
593 	- f(i,0,n): i -> SNK com valor Ai
9eb 	- f(i,0,m):
9e2 		* SRC -> n+i com valor Wi
a9e 		* ParaTodo dependente Bj: n+i -> Bj com peso INF
8a0 	- ans = somatorio(Wi) - maxFlow(SRC,SNK);

/* ======================================================= */
\end{lstlisting}

\subsection{IsBipartido}
\begin{lstlisting}
// Verifica se um grafo eh bipartido
// O(V+E)

b94 const int MAXN = 110;

ea6 vector<int> adj[MAXN];
496 int color[MAXN];

64d bool bipartido(int n) {
    
37f     int s = 0;
8b2     queue<int> q; q.push(s);
56d     color[s] = 0;
4fd     bool ans = true; 
    
984     while (!q.empty() and ans) { 
be1         int u = q.front(); q.pop();
    
cab         for (auto &v : adj[u]) {
f75             if (color[v] == INF) {                            
23d                 color[v] = 1 - color[u]; 
2a1                 q.push(v);
763             }
ec1             else if (color[v] == color[u]) {                        
7bd                 ans = false; 
c2b                 break;               
d89             }
d57         }
8fe     }
    
ba7     return ans;
61e }

5a4 void solve(int n) {
    
418 	f(i,0,n) {
9b0 		adj[i] = vector<int>();
f49 		color[i] = INF;
417 	}
    
    	// preenche grafo ...
    
bcc 	bool ans = bipartido(n);
b1b }
\end{lstlisting}

\subsection{Kruskal}
\begin{lstlisting}
// Encontra a Arvore Geradora Minima (AGM) de um grafo
// O(E log V)

da5 const int MAXN = 1e3+10;

320 int id[MAXN], sz[MAXN];

aee int find(int a) { 
3da     return id[a] = (id[a] == a ? a : find(id[a]));
c7b }

78b void uni(int a, int b) { // O(a(N)) amortizado
605     a = find(a), b = find(b);
d54     if(a == b) return;
    
3c6     if(sz[a] > sz[b]) swap(a,b);
6eb     id[a] = b, sz[b] += sz[a];
2de }

057 int kruskal(vector<tuple<int, int, int>>& edg) {
    	
704 	int cost = 0;
    	// vector<tuple<int, int, int>> mst;
fea 	for (auto [w,x,y] : edg) if (find(x) != find(y)) {
    		// mst.emplace_back(w, x, y);
45f 		cost += w; 
cf2 		uni(x,y);
815 	}
12d 	return cost;
798 }

e03 void solve(int n, int ed) {
    
f51 	vector<tuple<int, int, int>> edg(n);
    
863 	for(auto& [w,u,v] : edg) {
261 		cin >> u >> v >> w; u--, v--;
afd 	}
    
418 	f(i,0,n) {
963 		id[i] = i;
bc4 		sz[i] = -1;
55b 	}
    	
c14 	sort(all(edg));
    
772 	int cost = kuskal(edg);
8c8 }

// VARIANTES

// Maximum Spanning Tree: sort(edg.rbegin(), edg.rend());

// 'Minimum' Spanning Subgraph:
//	- Algumas arestas ja foram adicionadas (maior prioridade - Questao das rodovias)
//	- Arestas que nao foram adicionadas (menor prioridade - ferrovias)
//	-> kruskal(rodovias); kruskal(ferrovias);

// Minimum Spanning Forest:
// 	- Queremos uma floresta com k componentes
// 	-> kruskal(edg); if(mst.sizer() == k) break;

//  MiniMax
//	- Encontrar menor caminho entre dous vertices com maior quantidade de arestas
//	-> kruskal(edg); dijsktra(mst);

// Second Best MST
// 	- Encontrar a segunda melhor arvore geradora minima
// 	-> kruskal(edg);
// 	-> flag mst[i] = 1;
// 	-> sort(cmp(edg.flag != -1)) => da prioridade para outras arestas
\end{lstlisting}

\subsection{LCA com RMQ}
\begin{lstlisting}
// Assume que um vertice eh ancestral dele mesmo, ou seja,
// se a eh ancestral de b, lca(a, b) = a
// dist(a, b) retorna a distancia entre a e b
//
// build - O(n)
// lca - O(1)
// dist - O(1)

67a template<typename T> 
9f6 struct rmq {
517     vector<T> v;
1a8     int n; 
bac     static const int b = 30;
70e     vector<int> mask, t;
    
6b4     int op(int x, int y) { 
ffd         return v[x] < v[y] ? x : y; 
18e     }
543     int msb(int x) { 
391         return __builtin_clz(1) - __builtin_clz(x); 
ee1     }
6ad     rmq() {}
43c     rmq(const vector<T>& v_) : v(v_), n(v.size()), mask(n), t(n) {
2e5         for (int i = 0, at = 0; i < n; mask[i++] = at |= 1) {
a61             at = (at << 1) & ((1 << b) - 1);
411             while (at and op(i, i - msb(at & -at)) == i)
282                 at ^= at & -at;
53c         }
9c0         for (int i = 0; i < n / b; i++) 
e78             t[i] = b * i + b - 1 - msb(mask[b * i + b - 1]);
dce         for (int j = 1; (1 << j) <= n / b; j++) 
122             for (int i = 0; i + (1 << j) <= n / b; i++)
ba5                 t[n / b * j + i] = op(t[n / b * (j - 1) + i], t[n / b * (j - 1) + i + (1 << (j - 1))]);
2d3     }
879     int small(int r, int sz = b) { 
7e3         return r - msb(mask[r] & ((1 << sz) - 1)); 
c92     }
b7a     T query(int l, int r) {
cb5         if (r - l + 1 <= b)
e60             return small(r, r - l + 1);
7bf         int ans = op(small(l + b - 1), small(r));
e80         int x = l / b + 1, y = r / b - 1;
e25         if (x <= y) {
a4e             int j = msb(y - x + 1);
002             ans = op(ans, op(t[n / b * j + x], t[n / b * j + y - (1 << j) + 1]));
4b6         }
ba7         return ans;
6bf     }
b75 };

065 namespace lca {
282     vector<int> g[MAXN];
1d9     int v[2 * MAXN], pos[MAXN], dep[2 * MAXN];
8bd     int t;
2de     rmq<int> RMQ;
        
4cf     void dfs(int i, int d = 0, int p = -1) {
ae8         v[t] = i;
1f1         pos[i] = t;
949         dep[t] = d;
c82         t++;
45a         for (auto j : g[i])
f25             if(j != p) {
8ec                 dfs(j, d + 1, i);
ae8                 v[t] = i;
949                 dep[t] = d;
c82                 t++;
fcd             }
8de     }
        
789     void build(int n, int root) {
a34         t = 0;
14e         dfs(root);
659         vector<int> depVec(dep, dep + 2 * n - 1);
ac6         RMQ = rmq<int>(depVec);
631     }
        
7be     int lca(int a, int b) {
ab7         a = pos[a], b = pos[b];
d11         if(a > b) swap(a, b);
544         return v[RMQ.query(a, b)];
413     }
        
b5d     int dist(int a, int b) {
72b         int l = lca(a, b);
798         return dep[pos[a]] + dep[pos[b]] - 2 * dep[pos[l]];
2ed     }
767 }

   
5a4 void solve(int n) {
        
742 	f(i,0,n-1){
bf7         int a, b; cin >> a >> b; a--; b--;
ec9         lca::g[a].push_back(b);
b2f         lca::g[b].push_back(a);
32a     }
        
a78     lca::build(n, 0); // raiz nesse caso eh 0
      
d00 	cout << lca::dist(1,2) << endl // distancia entre vertices 1 e 2 da arvore
83a }

\end{lstlisting}

\subsection{Lowest Common Ancestor (LCA) com peso}
\begin{lstlisting}
// Encontra o LCA de uma arvore com peso, assim como a distancia 
// entre 2 vertices.
//
// Assume que um vertice eh ancestral dele mesmo, ou seja,
// se a eh ancestral de b, lca(a, b) = a
// dist(a, b) retorna a distancia entre a e b
//
// build - O(n)
// lca - O(1)
// dist - O(1)

47e const int MAXN = 1e5+10;

67a template<typename T> 
9f6 struct rmq {
517     vector<T> v;
1a8     int n; 
bac     static const int b = 30;
70e     vector<int> mask, t;
    
18e     int op(int x, int y) { return v[x] < v[y] ? x : y; }
ee1     int msb(int x) { return __builtin_clz(1) - __builtin_clz(x); }
        
6ad     rmq() {}
43c     rmq(const vector<T>& v_) : v(v_), n(v.size()), mask(n), t(n) {
2e5         for (int i = 0, at = 0; i < n; mask[i++] = at |= 1) {
a61             at = (at << 1) & ((1 << b) - 1);
411             while (at and op(i, i - msb(at & -at)) == i)
282                 at ^= at & -at;
53c         }
9c0         for (int i = 0; i < n / b; i++) 
e78             t[i] = b * i + b - 1 - msb(mask[b * i + b - 1]);
dce         for (int j = 1; (1 << j) <= n / b; j++) 
122             for (int i = 0; i + (1 << j) <= n / b; i++)
0ee                 t[n / b * j + i] = op(t[n / b * (j - 1) + i],
cc8                                        t[n / b * (j - 1) + i + (1 << (j - 1))]);
2d3     }
        
879     int small(int r, int sz = b) { 
7e3         return r - msb(mask[r] & ((1 << sz) - 1)); 
c92     }
        
b7a     T query(int l, int r) {
27b         if (r - l + 1 <= b) return small(r, r - l + 1);
7bf         int ans = op(small(l + b - 1), small(r));
e80         int x = l / b + 1, y = r / b - 1;
e25         if (x <= y) {
a4e             int j = msb(y - x + 1);
002             ans = op(ans, op(t[n / b * j + x], t[n / b * j + y - (1 << j) + 1]));
4b6         }
ba7         return ans;
6bf     }
b75 };

065 namespace lca {
2f8     vector<pair<int,int>> g[MAXN];
        
13c     int v[2 * MAXN];
e3b     int pos[MAXN];
9bb     int level[2 * MAXN];
8bd     int t;
2de     rmq<int> RMQ;
        
9d3     int dist[MAXN];
    
2f0     void dfs(int i, int l = 0, int p = -1, long long d = 0) {
ae8         v[t] = i;
1f1         pos[i] = t;
f68         level[t] = l;
0f9         dist[i] = d;
c82         t++;
eaf         for (auto edge : g[i]) {
149             int nxt = edge.first;
68a             int w = edge.second;
40e             if (nxt == p) continue;
749             dfs(nxt, l + 1, i, d + w);
ae8             v[t] = i;
f68             level[t] = l;
c82             t++;
001         }
165     }
        
cda     void build(int n, int root = 0) {
a34         t = 0;
14e         dfs(root);
c64         vector<int> levelVec(level, level + (2 * n - 1));
a0c         RMQ = rmq<int>(levelVec);
d91     }
        
7be     int lca(int a, int b) {
ab7         a = pos[a], b = pos[b];
d11         if (a > b) swap(a, b);
544         return v[RMQ.query(a, b)];
413     }
        
7a4     long long queryDist(int a, int b) {
851         int anc = lca(a, b);
88b         return dist[a] + dist[b] - 2LL * dist[anc];
731     }
c13 }

63d void solve() {
    	
9ee 	int n; cin >> n;
    
a45 	f(i,0,n)
5af 		lca::g[i].clear();
    	
8b2 	f(i,1,n) {
d25 		int a, b, w; cin >> a >> b >> w;
cbc 		lca::g[a].push_back({b, w});
b41 		lca::g[b].push_back({a, w});
d6c 	}
    	
a78 	lca::build(n, 0); // arvore com n vertices com raiz em 0
    
903 	int lowestCommonAncertor = lca::lca(0,1); // LCA entre 0 e 1
258 	int dist = lca::queryDist(0,1); // Distancia entre 0 e 1
    
df1 }
\end{lstlisting}

\subsection{Pontos de Articulacao + Pontes}
\begin{lstlisting}
// Computa os pontos de articulacao (vertices criticos) de um grafo
//
// art[i] armazena o numero de novas componentes criadas ao deletar vertice i
// se art[i] >= 1, entao vertice i eh ponto de articulacao
// 
// O(V + E)

aec const int MAXN = 410;

ea6 vector<int> adj[MAXN];
5d0 int id[MAXN], art[MAXN];
4ce stack<int> s;

3e1 int dfs_art(int i, int &t, int p = -1) {
cf0     int lo = id[i] = t++;
e07     int children = 0;
18e     s.push(i);
f78     for (int j : adj[i]) {
d09         if (j == p) continue;
9a3         if (id[j] == -1) {
c5f             children++;
206             int val = dfs_art(j, t, i);
0c3             lo = min(lo, val);
588             if (val >= id[i]) {
66a                 art[i]++;
bd9                 while (s.top() != j) s.pop();
2eb                 s.pop();
1f3             }
    			// if (val > id[i]) aresta i-j eh ponte
682         }
4e6         else {
872             lo = min(lo, id[j]);
30c         }
798     }
        
924 	if (p == -1) {
0d6         if (children > 1)
4f4             art[i] = children - 1;
295         else
2b9             art[i] = -1;
abc     }
253     return lo;
db5 }

4d9 void AP(int n) {
    
79e 	s = stack<int>();
        
418 	f(i,0,n) { 
9c8         id[i] = -1; 
2b9         art[i] = -1; 
ec6     }
6bb     int t = 0;
418     f(i,0,n) {
766         if (id[i] == -1)
625             dfs_art(i, t, -1);
d39     }
f67 }

e03 void solve(int n, int ed) {
    
98f 	int n, ed;
a45 	f(i,0,n)
9b0         adj[i] = vector<int>();
    
c92     while(ed--) {
ba2         int a, b; 
fab         adj[a].push_back(b);
c87         adj[b].push_back(a);
9a0     }
    
bd2     AP(n);
        
516     vector<int> pontos;
        // Para vertices nao-raiz, art[i] >= 0 indica que eh ponto de articulacao.
        // Para a raiz (i==0) ela so deve ser considerada se tiver 2 ou mais filhos, ou seja, se art[0] > 0.
418     f(i,0,n) {
995         if (i == 0) {
0f8             if (art[i] > 0) pontos.push_back(i+1);
e74         } else {
72a             if (art[i] >= 0) pontos.push_back(i+1);
802         }
c23     }
fdb }

\end{lstlisting}

\subsection{SSSP - Bellman Ford}
\begin{lstlisting}
// Aceita pesos negativos
//
// Conexo: O(VE)
// Desconexo: O(EV^2)

1a0 const int MAXEDG = 1e3+10;

203 tuple<int,int,int>> edg [MAXN];
989 int dist[MAXN];i

f6e int bellman_ford(int n, int src) {
11b     dist.assign(n+1, INT_MAX);
    
41e     f(i,0,n+2) {
d77         for(auto& [u, v, w] : edg) {
20a             if(dist[u] != INT_MAX and dist[v] > w + dist[u])
491                 dist[v] = dist[u] + w;
224         }
a1e     }
    
        // Possivel checar ciclos negativos (ciclo de peso total negativo)
d77     for(auto& [u, v, w] : edg) {
20a         if(dist[u] != INT_MAX and dist[v] > w + dist[u])
6a5             return 1;
40f     }
    
bb3     return 0;
0b4 }

e03 void solve(int n, int ed) {
    
040 	f(i,0,n) { dist[i] = INF; }
    
31c 	f(i,0,ed) {
2ef         int u, v, w; u--, v--;
732 		edg[i] = {u,v,w};
5ce     }
    
447     bellman_ford(n, 0);
bd8 }
\end{lstlisting}

\subsection{SSSP - Dijktra}
\begin{lstlisting}
// O(E log V)

2a7 const int MAXN = 1e4+10;

3ac vector<pair<int,int>> adj[MAXN];
9d3 int dist[MAXN];
// int parent[MAXN];

3f4 void dijkstra(int s) {
a93     dist[s] = 0; // se eventualmente puder voltar pra ca, tipo ciclo | salesman | remover essa linha
63c     priority_queue<pii, vector<pii>, greater<pii>> pq; pq.push({0, s});
    
502     while (!pq.empty()) {
5c1         auto [d, u] = pq.top(); pq.pop();
3e1         if (d > dist[u]) continue;
    		 // if(u == s and dist[u] < INF) break; | pra quando tiver que fazer um ciclo
    
3c0         for (auto &[v, w] : adj[u]) {
c21             if (dist[u] + w >= dist[v]) continue;
491             dist[v] = dist[u]+w;
    			// parent[v] = u;
bf3             pq.push({dist[v], v});
a42         }
6df     }
695 }

63d void solve() {
    
8ed 	int v, ed; cin >> v >> ed;
98b 	f(i,0,v) { 	dist[i] = INF; }
    
c92 	while(ed--) {
691 		int a, b, w; cin >> a >> b >> w; a--, b--;
fbc 		adj[a].emplace_back(b,w);
    		// adj[b].emplace_back(a,w);
47c 	}
458 	int s; dijkstra(s);
c49 }
\end{lstlisting}



%%%%%%%%%%%%%%%%%%%%
%
% Matematica
%
%%%%%%%%%%%%%%%%%%%%

\section{Matematica}

\subsection{Conversao de Bases}
\begin{lstlisting}
// Converte de 10 -> {2, 8, 10, 16} (log n)
// Converte de {2, 8, 10, 16} -> 10 (n)
9c7 char charForDigit(int digit) {
431     if (digit > 9) return digit + 87;
d4a     return digit + 48;
826 }

// 10 -> {2, 8, 10, 16}
0f3 string decimalToBase(int n, int base = 10) {
f40     if (not n) return "0";
461     stringstream ss;
fcb     for (int i = n; i > 0; i /= base) {
ac7         ss << charForDigit(i % base);
cd2     }
f1f     string s = ss.str();
01f     reverse(s.begin(), s.end());
047     return s;
01e }

9a2 int intForDigit(char digit) {
374     int intDigit = digit - 48;
545     if (intDigit > 9) return digit - 87;
a09     return intDigit;
acc }

// {2, 8, 10, 16} -> 10
e37 int baseToDecimal(const string& n, int base = 10) {
c18     int result = 0;
e09     int basePow =1;
000     for (auto it = n.rbegin(); it != n.rend(); ++it, basePow *= base)
445         result += intForDigit(*it) * basePow;
dc8     return result;
9f0 }
\end{lstlisting}

\subsection{Divisores - Contar}
\begin{lstlisting}
// Conta o numero de divisores de um numero baseadp no Smallest Prime Factor

191 vector<int> spf; // Smallest Prime Factor

254 void computeSpf(int n) {
768     spf.resize(n + 1);
78a     for (int i = 1; i <= n; i++) {
cdc         spf[i] = i;
7a5     }
2ed     for (int i = 2; i * i <= n; i++) {
a58         if (spf[i] == i) {
985             for (int j = i * i; j <= n; j += i) {
d91                 if (spf[j] == j)
9a0                     spf[j] = i;
622             }
44b         }
1ee     }
0fe }

1e3 int getDivisorCount(int x) {
4e4     int cntDiv = 1;
40e     while (x > 1) {
80a         int p = spf[x];
ac9         int cnt = 0;
fa7         while (x % p == 0) {
f65             cnt++;
43f             x /= p;
cfd         }
2ba         cntDiv *= (cnt + 1);
646     }
a87     return cntDiv;
d96 }

63d void solve() {
1a8 	int n; // maior dos numeros a ser computado a listagem
aee 	computeSpf(n); // gera os spf para todos ate n
d9c 	cout << getDivisorCount(n) << endl;
4f6 }
\end{lstlisting}

\subsection{MDC e MMC}
\begin{lstlisting}
// O(log n)

// MDC entre 2 numeros
8c1 int mdc(int a, int b) {
ce2     for (int r = a % b; r; a = b, b = r, r = a % b);
73f     return b;
8f5 }

// MDC entre N numeros
460 int mdc_many(vector<int> arr) {
7b6    int result = arr[0];
    
aa6    for (int& num : arr) {
437        result = mdc(num, result);
    
c03        if(result == 1) return 1;
885    }
dc8    return result;
0c9 }

// MMC entre 2 numeros
3ec int mmc(int a, int b) {
6fe     return a / mdc(a, b) * b;
770 }

// MMC entre N numeros
1db int mmc_many(vector<int> arr) {
7b6     int result = arr[0];
    
05f     for (int &num : arr)
9c4         result = (num * result / mdc(num, result));
dc8     return result;
72c }
\end{lstlisting}

\subsection{Numero de Digitos}
\begin{lstlisting}
// Calcula o numero de digitos de n 
// 1234 = 4; 0 = 1

09c int numDigits(int n) {
209     if (n == 0) return 1;
662     n = std::abs(n);
146     return static_cast<int>(std::floor(std::log10(n))) + 1;
d2b }
\end{lstlisting}

\subsection{Primos - Lowest Prime Factor}
\begin{lstlisting}
// Menor fator primo de n
// O(sqrt(n))

074 int lowestPrimeFactor(int n, int startPrime = 2) {
9d5     if (startPrime <= 3) {
fb4         if (not (n & 1)) return 2;
5a0         if (not (n % 3)) return 3;
72a         startPrime = 5;
43a     }
    
c94     for (int i = startPrime; i * i <= n; i += (i + 1) % 6 ? 4 : 2)
dcb         if (not (n % i))
d9a             return i;
041     return n;
6c5 }
\end{lstlisting}

\subsection{Primos - Primo}
\begin{lstlisting}
// Verifica se um numero eh primo
// O(sqrt(n))
5f7 bool isPrime(int n) {
e32     return n > 1 and lowestPrimeFactor(n) == n;
822 }
\end{lstlisting}

\subsection{Sieve}
\begin{lstlisting}
// Gera todos os primos do intervalo [1,lim]
// O(n log log n)

324 int _sieve_size;
467 bitset<10000010> bs;
632 vector<int> p;

5c4 void sieve(int lim) {
1ba     _sieve_size = lim+1;
0a3     bs.set();
e52     bs[0] = bs[1] = 0;
a0a     f(i,2,_sieve_size) { 
a47         if (bs[i]) {
bfe             for (int j = i*i; j < _sieve_size; j += i) bs[j] = 0;
d8d             p.push_back(i);
70b         }
ab8     }
841 }
\end{lstlisting}



%%%%%%%%%%%%%%%%%%%%
%
% String
%
%%%%%%%%%%%%%%%%%%%%

\section{String}

\subsection{Longest Common Subsequence 1 (LCS)}
\begin{lstlisting}
// Retrorna a LCS entre as string S e T.
// Armazena em memo[i][j] o LCS_SZ de s[i...n] e t[j...m].
// Implementacao recursiva
//
// Temporal: O(n*m)
// Espacial: O(n*m)

da5 const int MAXN = 1e3+10;

dd0 int memo[MAXN][MAXN];

// Calcula tamanho do LCS recursivamente
28f int lcs_sz(string& s, string& t, int i, int j) {
45d 	if(i == s.size() or j == t.size()) return 0;
e80 	auto& ans = memo[i][j];
d64 	if(~ans) return ans;
1a9 	if(s[i] == t[j])
176 		ans = 1 + lcs_sz(s,t,i+1, j+1);
295 	else
3af 		ans = max(
c19 				lcs_sz(s,t,i+1,j),
364 				lcs_sz(s,t,i,j+1)
616 			);
ba7 	return ans;
afa }

// Armazena em ans a LCS entre S e T
10e void lcs(string& ans, string& s, string& t, int i, int j) {
f86 	if(i >= s.size() or j >= t.size()) return;
524 	if(s[i] == t[j]) {
b80 		ans.push_back(s[i]);
081 		return lcs(ans, s, t, i+1, j+1);
a00 	}
4cb 	if(lcs_sz(s,t,i+1,j) > lcs_sz(s,t,i,j+1)) return lcs(ans, s, t, i+1, j);
4f2 	return lcs(ans, s, t, i, j+1);
bc5 }

63d void solve() {
    
bfb 	string s, t; cin >> s >> t;
    	
457 	memset(memo,-1, sizeof memo);
    
a4d 	string ans; lcs(ans, s, t, 0,0);
886 	cout << ans << endl;
030 }

\end{lstlisting}

\subsection{Split de String}
\begin{lstlisting}
// O(|s| * |del|).
5a6 vector<string> split(string s, string del = " ") {
cd5    vector<string> retorno;
0f4    int start, end = -1*del.size();
016    do {
a3b        start = end + del.size();
257        end = s.find(del, start);
36c        retorno.push_back(s.substr(start, end - start));
3a7    } while (end != -1);
5fa    return retorno;
f80 }
\end{lstlisting}

\pagebreak


%%%%%%%%%%%%%%%%%%%%
%
% Extra
%
%%%%%%%%%%%%%%%%%%%%

\section{Extra}

\subsection{fastIO.cpp}
\begin{lstlisting}
int read_int() {
    bool minus = false;
    int result = 0;
    char ch;
    ch = getchar();
    while (1) {
        if (ch == '-') break;
        if (ch >= '0' && ch <= '9') break;
        ch = getchar();
    }
    if (ch == '-') minus = true;
    else result = ch-'0';
    while (1) {
        ch = getchar();
        if (ch < '0' || ch > '9') break;
        result = result*10 + (ch - '0');
    }
    if (minus) return -result;
    else return result;
}
\end{lstlisting}

\subsection{hash.sh}
\begin{lstlisting}
# Para usar (hash das linhas [l1, l2]):
# bash hash.sh arquivo.cpp l1 l2
sed -n $2','$3' p' $1 | sed '/^#w/d' | cpp -dD -P -fpreprocessed | tr -d '[:space:]' | md5sum | cut -c-6
\end{lstlisting}

\subsection{stress.sh}
\begin{lstlisting}
P=a
make ${P} ${P}2 gen || exit 1
for ((i = 1; ; i++)) do
	./gen $i > in
	./${P} < in > out
	./${P}2 < in > out2
	if (! cmp -s out out2) then
		echo "--> entrada:"
		cat in
		echo "--> saida1:"
		cat out
		echo "--> saida2:"
		cat out2
		break;
	fi
	echo $i
done
\end{lstlisting}

\subsection{pragma.cpp}
\begin{lstlisting}
// Otimizacoes agressivas, pode deixar mais rapido ou mais devagar
#pragma GCC optimize("Ofast")
// Auto explicativo
#pragma GCC optimize("unroll-loops")
// Vetorizacao
#pragma GCC target("avx2")
// Para operacoes com bits
#pragma GCC target("bmi,bmi2,popcnt,lzcnt")
\end{lstlisting}

\subsection{timer.cpp}
\begin{lstlisting}
// timer T; T() -> retorna o tempo em ms desde que declarou
using namespace chrono;
struct timer : high_resolution_clock {
	const time_point start;
	timer(): start(now()) {}
	int operator()() {
		return duration_cast<milliseconds>(now() - start).count();
	}
};
\end{lstlisting}

\subsection{vimrc}
\begin{lstlisting}
189 "" {
d79 set ts=4 sw=4 mouse=a nu ai si undofile
7c9 function H(l)
496 	return system("sed '/^#/d' | cpp -dD -P -fpreprocessed | tr -d '[:space:]' | md5sum", a:l)
0be endfunction
329 function P() range
dd8 	for i in range(a:firstline, a:lastline)
ccc 		let l = getline(i)
139 		call cursor(i, len(l))
7c9 		echo H(getline(search('{}'[1], 'bc', i) ? searchpair('{', '', '}', 'bn') : i, i))[0:2] l
bf9 	endfor
0be endfunction
90e vmap <C-H> :call P()<CR>
de2 "" }
\end{lstlisting}

\subsection{debug.cpp}
\begin{lstlisting}
void debug_out(string s, int line) { cerr << endl; }
template<typename H, typename... T>
void debug_out(string s, int line, H h, T... t) {
    if (s[0] != ',') cerr << "Line(" << line << ") ";
    do { cerr << s[0]; s = s.substr(1);
    } while (s.size() and s[0] != ',');
    cerr << " = " << h;
    debug_out(s, line, t...);
}
#ifdef DEBUG
#define debug(...) debug_out(#__VA_ARGS__, __LINE__, __VA_ARGS__)
#else
#define debug(...) 42
#endif
\end{lstlisting}

\subsection{makefile}
\begin{lstlisting}
CXX = g++
CXXFLAGS = -fsanitize=address,undefined -fno-omit-frame-pointer -g -Wall -Wshadow -std=c++17 -Wno-unused-result -Wno-sign-compare -Wno-char-subscripts #-fuse-ld=gold

clearexe:
	find . -maxdepth 1 -type f -executable -exec rm {} +
\end{lstlisting}

\subsection{temp.cpp}
\begin{lstlisting}
#include <bits/stdc++.h>
using namespace std;

#define _ ios_base::sync_with_stdio(0);cin.tie(0);
#define all(a)       a.begin(), a.end()
#define int          long lo	ng int
#define double       long double
#define f(i,s,e) 	 for(int i=s;i<e;i++)
#define dbg(x) cout << #x << " = " << x << " ";
#define dbgl(x) cout << #x << " = " << x << endl;

#define vi 			 vector<int>
#define pii	         pair<int,int>
#define endl         "\n"
#define print_v(a)   for(auto x : a)cout<<x<<" ";cout<<endl
#define print_vp(a)  for(auto x : a)cout<<x.first<<" "<<x.second<< endl
#define rf(i,e,s) 	 for(int i=e-1;i>=s;i--)
#define CEIL(a, b)   ((a) + (b - 1))/b
#define TRUNC(x, n)  floor(x * pow(10, n))/pow(10, n)
#define ROUND(x, n)  round(x * pow(10, n))/pow(10, n)

const int INF = 1e9;    // 2^31-1
const int LLINF = 4e18; // 2^63-1
const double EPS = 1e-9;
const int MAX = 1e6+10; // 10^6 + 10

void solve() {

	
}

int32_t main() { _
	
	int t = 1; // cin >> t;
	while (t--) {
		solve();
	}
	return 0;
}
\end{lstlisting}

\subsection{rand.cpp}
\begin{lstlisting}
mt19937 rng((int) chrono::steady_clock::now().time_since_epoch().count());

int uniform(int l, int r){
	uniform_int_distribution<int> uid(l, r);
	return uid(rng);
}
\end{lstlisting}

\end{document}
